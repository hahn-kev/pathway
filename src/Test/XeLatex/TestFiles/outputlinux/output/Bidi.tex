\documentclass[a4paper]{article} 
\usepackage{float}
\usepackage{grffile}
\usepackage{graphicx}
\usepackage{amssymb}
\usepackage{fontspec}
\usepackage{fancyhdr}
\usepackage{multicol}
\usepackage{calc}
\usepackage{lettrine}
\usepackage{alphalph}
\usepackage[left=2cm,right=2cm,top=1.99cm,bottom=1.99cm,includeheadfoot]{geometry}
\usepackage{changepage}
\usepackage{needspace}
\usepackage{bidi} 
\usepackage{multicol}
\usepackage{setspace}

\parindent=0pt
\parskip=\medskipamount
\begin{document}
\pagestyle{plain}
\sloppy
\setlength{\parfillskip}{0pt plus 1fil}
\font\spanen="Times New Roman/GR" at 12pt
\font\spanur="Times New Roman/GR":color=ff0000 at 12pt
\font\diven="Times New Roman/GR" at 12pt
\font\divur="Times New Roman/GR" at 12pt
\font\xitemxitemdefinitionbefore="Times New Roman/GR" at 12pt
\font\xitemxitemexamplebefore="Charis SIL/GR/I" at 12pt
\font\xitemxitemexamplesbefore="Times New Roman/GR" at 12pt
\font\xitemxitemheadwordbefore="Charis SIL/GR/B" at 12pt
\font\xitemxitemheadwordminorbefore="Charis SIL/GR/B" at 12pt
\font\xitemxitemLexEntrypublishRootMinorPrimaryTargetHeadWordRefbefore="Charis SIL/GR/B" at 12pt
\font\xitemxitemlexreftargetsbefore="Times New Roman/GR" at 12pt
\font\letDatadicBody="Times New Roman/GR" at 12pt
\font\entryletDatadicBody="Times New Roman/GR" at 12pt
\font\headwordurentryletDatadicBody="Charis SIL/GR/B":color=ff0000 at 12pt
\font\xhomographnumberheadwordurentryletDatadicBody="Times New Roman/GR/B":color=ff0000 at 6.6pt
\font\spanenheadwordurentryletDatadicBody="Times New Roman/GR/B":color=ff0000 at 12pt
\font\sensesentryletDatadicBody="Times New Roman/GR" at 12pt
\font\sensesensesentryletDatadicBody="Times New Roman/GR" at 12pt
\font\grammaticalinfosensesensesentryletDatadicBody="Times New Roman/GR/I" at 12pt
\font\partofspeechengrammaticalinfosensesensesentryletDatadicBody="Times New Roman/GR/I" at 12pt
\font\spanenpartofspeechengrammaticalinfosensesensesentryletDatadicBody="Times New Roman/GR/I" at 12pt
\font\slotsgrammaticalinfosensesensesentryletDatadicBody="Times New Roman/GR/I" at 12pt
\font\spanenslotsgrammaticalinfosensesensesentryletDatadicBody="Times New Roman/GR/I" at 12pt
\font\slotnameenslotsgrammaticalinfosensesensesentryletDatadicBody="Times New Roman/GR/I" at 12pt
\font\spanenslotnameenslotsgrammaticalinfosensesensesentryletDatadicBody="Times New Roman/GR/I" at 12pt
\font\spanengrammaticalinfosensesensesentryletDatadicBody="Times New Roman/GR/I" at 12pt
\font\definitionensensesensesentryletDatadicBody="Times New Roman/GR" at 12pt
\font\spanendefinitionensensesensesentryletDatadicBody="Times New Roman/GR" at 12pt

\mbox{} 
\newpage 
\newpage 
\setcounter{page}{1} 
\pagenumbering{arabic} 
\pagestyle{fancy} 
\setlength{\columnsep}{1.5em} 
\setlength\columnseprule{0.4pt} 
\begin{multicols}{2}{\raggedleft} \begin{spacing}{1.5}
\hangindent= 12pt
\hangafter=1\RL{
\markboth{ \headwordurentryletDatadicBody -0}{ \headwordurentryletDatadicBody -0}\headwordurentryletDatadicBody{-0}\xhomographnumberheadwordurentryletDatadicBody{$_{1}$} } \spanenpartofspeechengrammaticalinfosensesensesentryletDatadicBody{Vst} \spanenslotsgrammaticalinfosensesensesentryletDatadicBody{: }\spanenslotnameenslotsgrammaticalinfosensesensesentryletDatadicBody{Asp} \spanendefinitionensensesensesentryletDatadicBody{PERF} \end{spacing}
 \end{multicols}
\end{document}
