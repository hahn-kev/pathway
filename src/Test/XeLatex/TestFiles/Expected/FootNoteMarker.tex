\documentclass[a4paper]{article} 
\usepackage{float}
\usepackage{grffile}
\usepackage{graphicx}
\usepackage{amssymb}
\usepackage{fontspec}
\usepackage{fancyhdr}
\usepackage{multicol}
\usepackage{calc}
\usepackage{lettrine}
\usepackage{alphalph}
\usepackage[left=1.83cm,right=1.83cm,top=1.51cm,bottom=1.51cm,includeheadfoot]{geometry}
\usepackage{changepage}
\usepackage{needspace}
\usepackage{multicol}
\usepackage{verbatim} 
\usepackage{setspace}

\parindent=0pt
\parskip=\medskipamount
\begin{document}
\pagestyle{plain}
\sloppy
\setlength{\parfillskip}{0pt plus 1fil}
\font\divzxx="Gautami/GR" at 12pt
\font\spanzxx="Gautami/GR" at 12pt
\font\picturepictureRight="Gautami/GR" at 12pt
\font\imgpicturedivpictureLeft="Gautami/GR" at 12pt
\font\scrBookscrBody="Gautami/GR" at 10pt
\font\scrBookNamezxxscrBookscrBody="Gautami/GR" at 10pt
\font\scrBookCodezxxscrBookscrBody="Gautami/GR" at 10pt
\font\TitleMainscrBookscrBody="Gautami/GR/B" at 20pt
\font\spanzxxTitleMainscrBookscrBody="Gautami/GR/B" at 20pt
\font\TitleSecondaryzxxTitleMainscrBookscrBody="Gautami/GR/B" at 20pt
\font\columnsscrBookscrBody="Gautami/GR" at 10pt
\font\scrSectioncolumnsscrBookscrBody="Gautami/GR" at 10pt
\font\SectionHeadscrSectioncolumnsscrBookscrBody="Gautami/GR/B" at 9pt
\font\spanzxxSectionHeadscrSectioncolumnsscrBookscrBody="Gautami/GR/B" at 9pt
\font\ParallelPassageReferencescrSectioncolumnsscrBookscrBody="Gautami/GR/I" at 10pt
\font\spanzxxParallelPassageReferencescrSectioncolumnsscrBookscrBody="Gautami/GR/I" at 10pt
\font\ParagraphscrSectioncolumnsscrBookscrBody="Gautami/GR" at 10pt
\font\ChapterNumberzxxParagraphscrSectioncolumnsscrBookscrBody="Gautami/GR/B" at 25pt
\font\VerseNumberzxxParagraphscrSectioncolumnsscrBookscrBody="Gautami/GR" at 6pt
\font\spanzxxParagraphscrSectioncolumnsscrBookscrBody="Gautami/GR" at 10pt
\font\liscrSectioncolumnsscrBookscrBody="Gautami/GR" at 10pt
\font\spanzxxliscrSectioncolumnsscrBookscrBody="Gautami/GR" at 10pt
\font\VerseNumberzxxspanzxxliscrSectioncolumnsscrBookscrBody="Gautami/GR" at 6pt
\font\spanzxxspanzxxliscrSectioncolumnsscrBookscrBody="Gautami/GR" at 10pt
\font\scrFootnoteMarkerParagraphscrSectioncolumnsscrBookscrBody="Gautami/GR" at 10pt
\font\NoteCrossHYPHENReferenceParagraphParagraphscrSectioncolumnsscrBookscrBody="Gautami/GR" at 8pt
\font\NoteTargetReferencezxxNoteCrossHYPHENReferenceParagraphParagraphscrSectioncolumnsscrBookscrBody="Gautami/GR" at 8pt
\font\spanzxxNoteCrossHYPHENReferenceParagraphParagraphscrSectioncolumnsscrBookscrBody="Gautami/GR" at 8pt
\font\NoteGeneralParagraphParagraphscrSectioncolumnsscrBookscrBody="Gautami/GR" at 8pt
\font\NoteTargetReferencezxxNoteGeneralParagraphParagraphscrSectioncolumnsscrBookscrBody="Gautami/GR" at 8pt
\font\ReferencedTextzxxNoteGeneralParagraphParagraphscrSectioncolumnsscrBookscrBody="Gautami/GR" at 8pt
\font\spanzxxNoteGeneralParagraphParagraphscrSectioncolumnsscrBookscrBody="Gautami/GR" at 8pt
\font\LinebscrSectioncolumnsscrBookscrBody="Gautami/GR" at 10pt
\font\VerseNumberzxxLinebscrSectioncolumnsscrBookscrBody="Gautami/GR" at 6pt
\font\spanzxxLinebscrSectioncolumnsscrBookscrBody="Gautami/GR" at 10pt
\font\LinecscrSectioncolumnsscrBookscrBody="Gautami/GR" at 10pt
\font\spanzxxLinecscrSectioncolumnsscrBookscrBody="Gautami/GR" at 10pt
\font\rqzxxLinecscrSectioncolumnsscrBookscrBody="Gautami/GR" at 10pt

\mbox{} 
\newpage 
\newpage 
\setcounter{page}{1} 
\pagenumbering{arabic} 
\pagestyle{fancy} 
\begin{comment}

\scrBookNamezxxscrBookscrBody{
 \label{PageStock_FootNoteMarker1} మత్తయి}\end{comment}
 
 \label{PageStock_FootNoteMarker1} \begin{comment}
\scrBookCodezxxscrBookscrBody{MAT}\end{comment}
 \begin{spacing}{1}\begin{center}\clubpenalty=300


\spanzxxTitleMainscrBookscrBody{మత్తయి }

\TitleSecondaryzxxTitleMainscrBookscrBody{సువార్త }
\end{center}\end{spacing}\setlength{\columnsep}{12pt} 
\setlength\columnseprule{0.4pt} 
\begin{multicols}{2}\begin{spacing}{0.5}\section*{\needspace {8\baselineskip}\begin{spacing}{0.5}\begin{center}
\spanzxxSectionHeadscrSectioncolumnsscrBookscrBody{యేసు వంశావళి }\end{center}\end{spacing}}
\begin{center}
\spanzxxParallelPassageReferencescrSectioncolumnsscrBookscrBody{(లూకా 3:23-38) }\end{center}
\begin{spacing}{0.5}\widowpenalty=300
\clubpenalty=300
\lettrine{
\ChapterNumberzxxParagraphscrSectioncolumnsscrBookscrBody{1}}{\VerseNumberzxxParagraphscrSectioncolumnsscrBookscrBody{$^{1}$}}\spanzxxParagraphscrSectioncolumnsscrBookscrBody{ యేసు క్రీస్తు వంశక్రమము: ఈయన దావీదు మరియు అబ్రాహాము వంశానికి చెందినవాడు. }\end{spacing}

\VerseNumberzxxspanzxxliscrSectioncolumnsscrBookscrBody{$^{2}$}\spanzxxspanzxxliscrSectioncolumnsscrBookscrBody{ అబ్రాహాము కుమారుడు ఇస్సాకు. }

\spanzxxspanzxxliscrSectioncolumnsscrBookscrBody{ఇస్సాకు కుమారుడు యాకోబు }

\spanzxxspanzxxliscrSectioncolumnsscrBookscrBody{యాకోబు కుమారులు యూదా మరియు అతని సహోదరులు. }

\VerseNumberzxxspanzxxliscrSectioncolumnsscrBookscrBody{$^{3}$}\spanzxxspanzxxliscrSectioncolumnsscrBookscrBody{ యూదా కుమారులు పెరెసు మరియు జెరహు. (పెరెసు, జెరహుల తల్లి తామారు.) }

\spanzxxspanzxxliscrSectioncolumnsscrBookscrBody{పెరెసు కుమారుడు ఎస్రోము }

\spanzxxspanzxxliscrSectioncolumnsscrBookscrBody{ఎస్రోము కుమారుడు అరాము }

\VerseNumberzxxspanzxxliscrSectioncolumnsscrBookscrBody{$^{4}$}\spanzxxspanzxxliscrSectioncolumnsscrBookscrBody{ అరాము కుమారుడు అమ్మీనాదాబు. }

\spanzxxspanzxxliscrSectioncolumnsscrBookscrBody{అమ్మీనాదాబు కుమారుడు నయస్సోను. }

\spanzxxspanzxxliscrSectioncolumnsscrBookscrBody{నయస్సోను కుమారుడు శల్మా }

\VerseNumberzxxspanzxxliscrSectioncolumnsscrBookscrBody{$^{5}$}\spanzxxspanzxxliscrSectioncolumnsscrBookscrBody{ శల్మా కుమారుడు బోయజు. (బోయజు తల్లి రాహాబు.) }

\spanzxxspanzxxliscrSectioncolumnsscrBookscrBody{బోయజు కుమారుడు ఓబేదు. (ఓబేదు తల్లి రూతు.) }

\spanzxxspanzxxliscrSectioncolumnsscrBookscrBody{ఓబేదు కుమారుడు యెష్షయి. }

\VerseNumberzxxspanzxxliscrSectioncolumnsscrBookscrBody{$^{6}$}\spanzxxspanzxxliscrSectioncolumnsscrBookscrBody{ యెష్షయి కుమారుడు రాజు దావీదు. }

\spanzxxspanzxxliscrSectioncolumnsscrBookscrBody{దావీదు కుమారుడు సొలొమోను. (సొలొమోను తల్లి పూర్వం ఊరియా భార్య.) }

\VerseNumberzxxspanzxxliscrSectioncolumnsscrBookscrBody{$^{7}$}\spanzxxspanzxxliscrSectioncolumnsscrBookscrBody{ సొలొమోను కుమారుడు రెహబాము. }

\spanzxxspanzxxliscrSectioncolumnsscrBookscrBody{రెహబాము కుమారుడు అబీయా. }

\spanzxxspanzxxliscrSectioncolumnsscrBookscrBody{అబీయా కుమారుడు ఆసా. }

\VerseNumberzxxspanzxxliscrSectioncolumnsscrBookscrBody{$^{8}$}\spanzxxspanzxxliscrSectioncolumnsscrBookscrBody{ ఆసా కుమారుడు యెహోషాపాతు. }

\spanzxxspanzxxliscrSectioncolumnsscrBookscrBody{యెహోషాపాతు కుమారుడు యెహోరాము. }

\spanzxxspanzxxliscrSectioncolumnsscrBookscrBody{యెహోరాము కుమారుడు ఉజ్జియా. }

\VerseNumberzxxspanzxxliscrSectioncolumnsscrBookscrBody{$^{9}$}\spanzxxspanzxxliscrSectioncolumnsscrBookscrBody{ ఉజ్జియా కుమారుడు యోతాము. }

\spanzxxspanzxxliscrSectioncolumnsscrBookscrBody{యోతాము కుమారుడు ఆహాజు. }

\spanzxxspanzxxliscrSectioncolumnsscrBookscrBody{ఆహాజు కుమారుడు హిజ్కియా. }

\VerseNumberzxxspanzxxliscrSectioncolumnsscrBookscrBody{$^{10}$}\spanzxxspanzxxliscrSectioncolumnsscrBookscrBody{ హిజ్కియా కుమారుడు మనష్షే. }

\spanzxxspanzxxliscrSectioncolumnsscrBookscrBody{మనష్షే కుమారుడు ఆమోసు. }

\spanzxxspanzxxliscrSectioncolumnsscrBookscrBody{ఆమోసు కుమారుడు యోషీయా. }

\VerseNumberzxxspanzxxliscrSectioncolumnsscrBookscrBody{$^{11}$}\spanzxxspanzxxliscrSectioncolumnsscrBookscrBody{ యోషీయా కుమారులు యెకొన్యా మరియు అతని సోదరులు. వీళ్ళ కాలంలోనే యూదులు బబులోను నగరానికి బందీలుగా కొనిపోబడినారు. }

\VerseNumberzxxspanzxxliscrSectioncolumnsscrBookscrBody{$^{12}$}\spanzxxspanzxxliscrSectioncolumnsscrBookscrBody{ బబులోను నగరానికి కొనిపోబడిన తరువాతి వంశ క్రమము: }

\spanzxxspanzxxliscrSectioncolumnsscrBookscrBody{యెకొన్యా కుమారుడు షయల్తీయేలు. }

\spanzxxspanzxxliscrSectioncolumnsscrBookscrBody{షయల్తీయేలు కుమారుడు జెరుబ్బాబెలు. }

\VerseNumberzxxspanzxxliscrSectioncolumnsscrBookscrBody{$^{13}$}\spanzxxspanzxxliscrSectioncolumnsscrBookscrBody{ జెరుబ్బాబెలు కుమారుడు అబీహూదు. }

\spanzxxspanzxxliscrSectioncolumnsscrBookscrBody{అబీహూదు కుమారుడు ఎల్యాకీము. }

\spanzxxspanzxxliscrSectioncolumnsscrBookscrBody{ఎల్యాకీము కుమారుడు అజోరు. }

\VerseNumberzxxspanzxxliscrSectioncolumnsscrBookscrBody{$^{14}$}\spanzxxspanzxxliscrSectioncolumnsscrBookscrBody{ అజోరు కుమారుడు సాదోకు. }

\spanzxxspanzxxliscrSectioncolumnsscrBookscrBody{సాదోకు కుమారుడు ఆకీము. }

\spanzxxspanzxxliscrSectioncolumnsscrBookscrBody{ఆకీము కుమారుడు ఎలీహూదు. }

\VerseNumberzxxspanzxxliscrSectioncolumnsscrBookscrBody{$^{15}$}\spanzxxspanzxxliscrSectioncolumnsscrBookscrBody{ ఎలీహూదు కుమారుడు ఎలియాజరు. }

\spanzxxspanzxxliscrSectioncolumnsscrBookscrBody{ఎలియాజరు కుమారుడు మత్తాను. }

\spanzxxspanzxxliscrSectioncolumnsscrBookscrBody{మత్తాను కుమారుడు యాకోబు. }

\VerseNumberzxxspanzxxliscrSectioncolumnsscrBookscrBody{$^{16}$}\spanzxxspanzxxliscrSectioncolumnsscrBookscrBody{ యాకోబు కుమారుడు యోసేపు. }

\spanzxxspanzxxliscrSectioncolumnsscrBookscrBody{యోసేపు భార్య మరియ. }

\spanzxxspanzxxliscrSectioncolumnsscrBookscrBody{మరియ కుమారుడు యేసు. ఈయన్ని క్రీస్తు అంటారు. }
\begin{spacing}{0.5}\widowpenalty=300
\clubpenalty=300

\VerseNumberzxxParagraphscrSectioncolumnsscrBookscrBody{$^{17}$}\spanzxxParagraphscrSectioncolumnsscrBookscrBody{ అంటే అబ్రాహాము కాలం నుండి దావీదు కాలం వరకు మొత్తం పదునాలుగు తరాలు. దావీదు కాలం నుండి బబులోను నగరానికి బందీలుగా కొనిపోబడిన కాలం వరకు పదునాలుగు తరాలు. అలా కొనిపోబడిన కాలం నుండి క్రీస్తు వరకు పదునాలుగు తరాలు. }\end{spacing}
\section*{\needspace {8\baselineskip}\begin{spacing}{0.5}\begin{center}
\spanzxxSectionHeadscrSectioncolumnsscrBookscrBody{యేసు క్రీస్తు జననం }\end{center}\end{spacing}}
\begin{center}
\spanzxxParallelPassageReferencescrSectioncolumnsscrBookscrBody{(లూకా 2:1-7) }\end{center}
\begin{spacing}{0.5}\widowpenalty=300
\clubpenalty=300

\VerseNumberzxxParagraphscrSectioncolumnsscrBookscrBody{$^{18}$}\spanzxxParagraphscrSectioncolumnsscrBookscrBody{ యేసు క్రీస్తు జననం ఇలా సంభవించింది: యేసు క్రీస్తు తల్లి మరియకు, యోసేపు అనే వ్యక్తికి వివాహం నిశ్చయమై ఉంది. వివాహంకాకముందే పవిత్రాత్మ శక్తి ద్వారా మరియ గర్భవతి అయింది. }\VerseNumberzxxParagraphscrSectioncolumnsscrBookscrBody{$^{19}$}\spanzxxParagraphscrSectioncolumnsscrBookscrBody{ కాని ఆమె భర్త యోసేపు నీతిమంతుడు. అందువల్ల అతడు అమెను నలుగురిలో అవమాన పరచదలచుకోలేదు. ఆమెతో రహస్యంగా తెగతెంపులు చేసుకోవాలని మనస్సులో అనుకొన్నాడు. }\end{spacing}
\begin{spacing}{0.5}\widowpenalty=300
\clubpenalty=300

\VerseNumberzxxParagraphscrSectioncolumnsscrBookscrBody{$^{20}$}\spanzxxParagraphscrSectioncolumnsscrBookscrBody{ అతడీవిధంగా అనుకొన్న తర్వాత, దేవదూత అతనికి కలలో కనిపించి, “యోసేపూ, దావీదు కుమారుడా, మరియ పవిత్రాత్మ ద్వారా గర్భవతి అయింది. కనుక ఆమెను భార్యగా స్వీకరించటానికి భయపడకు. }\VerseNumberzxxParagraphscrSectioncolumnsscrBookscrBody{$^{21}$}\spanzxxParagraphscrSectioncolumnsscrBookscrBody{ ఆమె ఒక మగ శిశువును ప్రసవిస్తుంది. ఆయన తన ప్రజల్ని వాళ్ళు చేసిన పాపాలనుండి రక్షిస్తాడు. కనుక ఆయనకు ‘యేసు’ అని పేరు పెట్టు” అని అన్నాడు. }\end{spacing}
\begin{spacing}{0.5}\widowpenalty=300
\clubpenalty=300

\VerseNumberzxxParagraphscrSectioncolumnsscrBookscrBody{$^{22-23}$}\spanzxxParagraphscrSectioncolumnsscrBookscrBody{ ప్రవక్త ద్వారా ప్రభువు ఈ విధంగా చెప్పాడు: “కన్యక గర్భవతియై మగ శిశువును ప్రసవిస్తుంది. వాళ్ళాయనను ఇమ్మానుయేలు అని పిలుస్తారు”}\footnote {\spanzxxNoteCrossHYPHENReferenceParagraphParagraphscrSectioncolumnsscrBookscrBody{1:22-23 ఉల్లేఖము: యెషయా 7:14.}}\spanzxxParagraphscrSectioncolumnsscrBookscrBody{ ఇది నిజం కావటానికే ఇలా జరిగింది. }\end{spacing}
\begin{spacing}{0.5}\widowpenalty=300
\clubpenalty=300

\VerseNumberzxxParagraphscrSectioncolumnsscrBookscrBody{$^{24}$}\spanzxxParagraphscrSectioncolumnsscrBookscrBody{ యోసేపు నిద్రలేచి దేవదూత ఆజ్ఞాపించినట్లు చేసాడు. మరియను తన భార్యగా స్వీకరించి తన ఇంటికి పిలుచుకు వెళ్ళాడు. }\VerseNumberzxxParagraphscrSectioncolumnsscrBookscrBody{$^{25}$}\spanzxxParagraphscrSectioncolumnsscrBookscrBody{ కాని, ఆమె కుమారుణ్ణి ప్రసవించే వరకు అతడు ఆమెతో కలియలేదు. అతడు ఆ బాలునికి “యేసు” అని నామకరణం చేసాడు. }\end{spacing}
\section*{\needspace {8\baselineskip}\begin{spacing}{0.5}\begin{center}
\spanzxxSectionHeadscrSectioncolumnsscrBookscrBody{తూర్పు నుండి జ్ఞానులు రావటం }\end{center}\end{spacing}}
\begin{spacing}{0.5}\widowpenalty=300
\clubpenalty=300
\lettrine{
\ChapterNumberzxxParagraphscrSectioncolumnsscrBookscrBody{2}}{\VerseNumberzxxParagraphscrSectioncolumnsscrBookscrBody{$^{1}$}}\spanzxxParagraphscrSectioncolumnsscrBookscrBody{ హేరోదు రాజ్యపాలన చేస్తున్న కాలంలో యూదయ దేశంలోని బేత్లెహేములో యేసు జన్మించాడు. తూర్పు దిశనుండి జ్ఞానులు యెరూషలేముకు వచ్చి }\VerseNumberzxxParagraphscrSectioncolumnsscrBookscrBody{$^{2}$}\spanzxxParagraphscrSectioncolumnsscrBookscrBody{ “యూదుల రాజుగా జన్మించినవాడు ఎక్కడున్నాడు? తూర్పున మేమాయన నక్షత్రాన్ని చూసి ఆయన్ని ఆరాధించటానికి వచ్చాము” అని అన్నారు. }\end{spacing}
\begin{spacing}{0.5}\widowpenalty=300
\clubpenalty=300

\VerseNumberzxxParagraphscrSectioncolumnsscrBookscrBody{$^{3}$}\spanzxxParagraphscrSectioncolumnsscrBookscrBody{ ఈ విషయం విని హేరోదు చాలా కలవరం చెందాడు. అతనితో పాటు యెరూషలేము ప్రజలు కూడ కలవరపడ్డారు. }\VerseNumberzxxParagraphscrSectioncolumnsscrBookscrBody{$^{4}$}\spanzxxParagraphscrSectioncolumnsscrBookscrBody{ అతడు ప్రధానయాజకుల్ని, పండితుల్ని}\footnote {\spanzxxNoteGeneralParagraphParagraphscrSectioncolumnsscrBookscrBody{2:4 పండితుల్ని లేక శాస్త్రుల్ని.}}\spanzxxParagraphscrSectioncolumnsscrBookscrBody{ సమావేశపరచి, “క్రీస్తు ఎక్కడ జన్మించబోతున్నాడు?” అని అడిగాడు. }\VerseNumberzxxParagraphscrSectioncolumnsscrBookscrBody{$^{5}$}\spanzxxParagraphscrSectioncolumnsscrBookscrBody{ వాళ్ళు, “యూదయ దేశంలోని బేత్లెహేములో” అని సమాధానం చెప్పారు. దీన్ని గురించి ప్రవక్త ఈ విధంగా వ్రాసాడు: }\end{spacing}
\begin{spacing}{0.5}
\hangindent= 24pt
\hangafter=1
\VerseNumberzxxLinebscrSectioncolumnsscrBookscrBody{$^{6}$}\spanzxxLinebscrSectioncolumnsscrBookscrBody{ “‘యూదయ దేశంలోని బేత్లెహేమా! } \end{spacing}
\begin{spacing}{0.5}
\hangindent= 24pt
\hangafter=1
\spanzxxLinecscrSectioncolumnsscrBookscrBody{నీవు యూదయ పాలకులకన్నా తక్కువేమీ కాదు! } \end{spacing}
\begin{spacing}{0.5}
\hangindent= 24pt
\hangafter=1
\spanzxxLinebscrSectioncolumnsscrBookscrBody{ఎందుకంటే, నీ నుండి ఒక పాలకుడు వస్తాడు. } \end{spacing}
\begin{spacing}{0.5}
\hangindent= 24pt
\hangafter=1
\spanzxxLinecscrSectioncolumnsscrBookscrBody{ఆయన నా ప్రజల, అంటే ఇశ్రాయేలు ప్రజల, కాపరిగా ఉంటాడు.’” }\rqzxxLinecscrSectioncolumnsscrBookscrBody{మీకా 5:2 } \end{spacing}
\begin{spacing}{0.5}\widowpenalty=300
\clubpenalty=300

\VerseNumberzxxParagraphscrSectioncolumnsscrBookscrBody{$^{7}$}\spanzxxParagraphscrSectioncolumnsscrBookscrBody{ ఆ తర్వాత హేరోదు జ్ఞానుల్ని రహస్యంగా పిలిచి ఆ నక్షత్రం కనిపించిన సరియైన సమయం వాళ్ళనడిగి తెలుసుకొన్నాడు. }\VerseNumberzxxParagraphscrSectioncolumnsscrBookscrBody{$^{8}$}\spanzxxParagraphscrSectioncolumnsscrBookscrBody{ వాళ్ళను బేత్లెహేముకు పంపుతూ, “వెళ్ళి, ఆ శిశువును గురించి సమాచారం పూర్తిగా కనుక్కోండి. ఆ శిశువును కనుక్కొన్నాక నాకు వచ్చి చెప్పండి. అప్పుడు నేను కూడా వచ్చి ఆరాధిస్తాను” అని అన్నాడు. }\end{spacing}
\begin{spacing}{0.5}\widowpenalty=300
\clubpenalty=300

\VerseNumberzxxParagraphscrSectioncolumnsscrBookscrBody{$^{9}$}\spanzxxParagraphscrSectioncolumnsscrBookscrBody{ వాళ్ళు రాజు మాటలు విని తమ దారిన తాము వెళ్ళిపొయ్యారు. వాళ్ళు తూర్పు దిశన చూసిన నక్షత్రం వాళ్ళకన్నా ముందు వెళ్ళుతూ ఆ శిశువు ఉన్న ఇంటి మీద ఆగింది. }\VerseNumberzxxParagraphscrSectioncolumnsscrBookscrBody{$^{10}$}\spanzxxParagraphscrSectioncolumnsscrBookscrBody{ వాళ్ళా నక్షత్రం ఆగిపోవటం చూసి చాలా ఆనందించారు. }\end{spacing}
\begin{spacing}{0.5}\widowpenalty=300
\clubpenalty=300

\VerseNumberzxxParagraphscrSectioncolumnsscrBookscrBody{$^{11}$}\spanzxxParagraphscrSectioncolumnsscrBookscrBody{ ఇంట్లోకి వెళ్ళి ఆ పసివాడు తన తల్లి మరియతో ఉండటం చూసారు. వాళ్ళు ఆయన ముందు మోకరిల్లి ఆయన్ని ఆరాధించారు. ఆ తర్వాత తమ కానుకల మూటలు విప్పి ఆయనకు బంగారు కానుకలు, సాంబ్రాణి, బోళం బహూకరించారు. }\VerseNumberzxxParagraphscrSectioncolumnsscrBookscrBody{$^{12}$}\spanzxxParagraphscrSectioncolumnsscrBookscrBody{ హేరోదు దగ్గరకు వెళ్ళొద్దని దేవుడు ఆ జ్ఞానులతో చెప్పాడు. అందువల్ల వాళ్ళు తమ దేశానికి మరో దారి మీదుగా వెళ్ళిపోయారు. }\end{spacing}
\section*{\needspace {8\baselineskip}\begin{spacing}{0.5}\begin{center}
\spanzxxSectionHeadscrSectioncolumnsscrBookscrBody{చివరి మాట }\end{center}\end{spacing}}
\begin{spacing}{0.5}\widowpenalty=300
\clubpenalty=300

\VerseNumberzxxParagraphscrSectioncolumnsscrBookscrBody{$^{13}$}\spanzxxParagraphscrSectioncolumnsscrBookscrBody{ దేవుని కుమారుని పేరులో విశ్వాసం ఉన్న మీకు నిత్యజీవం లభిస్తుంది. ఈ విషయం మీకు తెలియాలని యివన్నీ మీకు వ్రాస్తున్నాను. }\VerseNumberzxxParagraphscrSectioncolumnsscrBookscrBody{$^{14}$}\spanzxxParagraphscrSectioncolumnsscrBookscrBody{ దేవుణ్ణి ఆయన యిష్టానుసారంగా మనము ఏది అడిగినా వింటాడు. దేవుణ్ణి సమీపించటానికి మనకు హామీ ఉంది. }\VerseNumberzxxParagraphscrSectioncolumnsscrBookscrBody{$^{15}$}\spanzxxParagraphscrSectioncolumnsscrBookscrBody{ మనమేది అడిగినా వింటాడని మనకు తెలిస్తే మన మడిగింది మనకు లభించినట్లే కదా! }\end{spacing}
\begin{spacing}{0.5}\widowpenalty=300
\clubpenalty=300

\VerseNumberzxxParagraphscrSectioncolumnsscrBookscrBody{$^{16}$}\spanzxxParagraphscrSectioncolumnsscrBookscrBody{ మరణం కలిగించే పాపము తన సోదరుడు చెయ్యటం చూసిన వాడు తన సోదరుని కోసం దేవుణ్ణి ప్రార్థించాలి. అప్పుడు దేవుడు అతనికి క్రొత్త జీవితం యిస్తాడు. ఎవరి పాపం మరణానికి దారితీయదో వాళ్ళను గురించి నేను మాట్లాడుతున్నాను. మరణాన్ని కలిగించే పాపం విషయంలో ప్రార్థించమని నేను చెప్పటం లేదు. }\VerseNumberzxxParagraphscrSectioncolumnsscrBookscrBody{$^{17}$}\spanzxxParagraphscrSectioncolumnsscrBookscrBody{ ఏ తప్పు చేసినా పాపమే. కాని మరణానికి దారితీయని పాపాలు కూడా ఉన్నాయి. }\end{spacing}
\begin{spacing}{0.5}\widowpenalty=300
\clubpenalty=300

\VerseNumberzxxParagraphscrSectioncolumnsscrBookscrBody{$^{18}$}\spanzxxParagraphscrSectioncolumnsscrBookscrBody{ దేవుని బిడ్డగా జన్మించిన వాడు పాపం చెయ్యడని మనకు తెలుసు. తన బిడ్డగా జన్మించిన వాణ్ణి దేవుడు కాపాడుతాడు. సైతాను అతణ్ణి తాకలేడు. }\VerseNumberzxxParagraphscrSectioncolumnsscrBookscrBody{$^{19}$}\spanzxxParagraphscrSectioncolumnsscrBookscrBody{ మనము దేవుని సంతానమని, ప్రపంచమంతా సైతాను ఆధీనంలో ఉందని మనకు తెలుసు. }\VerseNumberzxxParagraphscrSectioncolumnsscrBookscrBody{$^{20}$}\spanzxxParagraphscrSectioncolumnsscrBookscrBody{ దేవుని కుమారుడు వచ్చి నిజమైన వాడెవడో తెలుసుకొనే జ్ఞానాన్ని మనకు యిచ్చాడు. ఇది మనకు తెలుసు. మనము నిజమైన వానిలో ఐక్యమై ఉన్నాము. ఆయన కుమారుడైన యేసు క్రీస్తులో కూడా ఐక్యమై ఉన్నాము. ఆయన నిజమైన దేవుడు. ఆయనే నిత్యజీవం. }\VerseNumberzxxParagraphscrSectioncolumnsscrBookscrBody{$^{21}$}\spanzxxParagraphscrSectioncolumnsscrBookscrBody{ బిడ్డలారా! విగ్రహాలకు దూరంగా ఉండండి. }\end{spacing}
\end{spacing}\end{multicols}
\end{document}
