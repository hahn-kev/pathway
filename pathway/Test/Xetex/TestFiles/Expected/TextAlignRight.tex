\font\a="Times New Roman" at 12pt
\rightline{\a{Right}
}

\bye
