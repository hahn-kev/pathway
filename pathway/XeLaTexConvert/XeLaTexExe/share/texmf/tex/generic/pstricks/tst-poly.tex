%%
%% This is file `tst-poly.tex',
%% generated with the docstrip utility.
%%
%% The original source files were:
%%
%% pst-poly.dtx  (with options: `tst-poly')
%% 
%% IMPORTANT NOTICE:
%% 
%% For the copyright see the source file.
%% 
%% Any modified versions of this file must be renamed
%% with new filenames distinct from tst-poly.tex.
%% 
%% For distribution of the original source see the terms
%% for copying and modification in the file pst-poly.dtx.
%% 
%% This generated file may be distributed as long as the
%% original source files, as listed above, are part of the
%% same distribution. (The sources need not necessarily be
%% in the same archive or directory.)
%%
%% Package `pst-poly.dtx'
%%
%% Denis Girou (CNRS/IDRIS - France) <Denis.Girou@idris.fr>
%% Herbert Voss <voss _at_ pstricks.de>
%%
%% Novermber 20, 2004
%%
%% This program can be redistributed and/or modified under the terms
%% of the LaTeX Project Public License Distributed from CTAN archives
%% in directory macros/latex/base/lppl.txt.
%%
%% DESCRIPTION:
%%   `pst-poly' is a PSTricks package to draw easily various kinds of regular
%%   or non regular polygons, with various customizations.
%%
%% `pst-poly' test file.
%%
\documentclass{article}

\usepackage{fancyvrb}
\usepackage[width=18cm]{geometry}
\usepackage{pstricks}
\usepackage{pst-poly}


\makeatletter


\def\HLEmphasize#1{\textit{#1}}
\newcommand{\BS}{\texttt{\symbol{`\\}}}
\def\HLMacro#1{\BS{}def\HLMacro@i#1\@nil}
\def\HLMacro@i#1def#2\@nil{\HLReverse{#2}}
\def\HLReverse#1{{\setlength{\fboxsep}{1pt}\HLReverse@i{#1}}}
\def\HLReverse@i#1{\colorbox{black}{\textcolor{white}{\textbf{#1}}}}

\def\Example{\FV@Environment{}{Example}}
\def\endExample{%
\end{VerbatimOut}
\Below@Example{\jobname.tmp}
\endgroup}

\def\CenterExample{\FV@Environment{}{Example}}
\def\endCenterExample{%
\end{VerbatimOut}
\begin{center}
\Below@Example{\jobname.tmp}
\end{center}
\endgroup}

\def\SideBySideExample{\FV@Environment{}{Example}}
\def\endSideBySideExample{%
\end{VerbatimOut}
\SideBySide@Example{\jobname.tmp}
\endgroup}

\def\FVB@Example{%
\begingroup
\FV@UseKeyValues
\parindent=0pt
\multiply\topsep by 2
\VerbatimEnvironment
\begin{VerbatimOut}[gobble=2,codes={\catcode`\Z=12}]{\jobname.tmp}}

\def\Below@Example#1{%
\VerbatimInput[commentchar=Z,commandchars=/?_,frame=single,
               numbers=left,numbersep=3pt]{#1}
\catcode`\%=14\relax
\catcode`\Z=9\relax
\catcode`/=0\relax
\catcode`?=1\relax
\catcode`_=2\relax
\def\HLEmphasize##1{##1}%
\def\HLMacro##1{##1}%
\def\HLReverse##1{##1}%
\input{#1}\par}

\def\SideBySide@Example#1{%
\vskip 1mm
\@tempdimb=\FV@XRightMargin
\advance\@tempdimb -5mm
\begin{minipage}[c]{\@tempdimb}
  \fvset{xrightmargin=0pt}
  \catcode`\%=14\relax
  \catcode`\Z=9\relax
  % We suppress the effect of the highlighting macros
  \catcode`/=0\relax
  \catcode`?=1\relax
  \catcode`_=2\relax
  \def\HLEmphasize##1{##1}%
  \def\HLMacro##1{##1}%
  \def\HLReverse##1{##1}%
  \input{#1}
\end{minipage}%
\@tempdimb=\textwidth
\advance\@tempdimb -\FV@XRightMargin
\advance\@tempdimb 5mm
\begin{minipage}[c]{\@tempdimb}
  \VerbatimInput[commentchar=Z,commandchars=/?_,
                 frame=single,numbers=left,numbersep=3pt,
                 xleftmargin=5mm,xrightmargin=0pt]{#1}
\end{minipage}
\vskip 1mm}

\makeatother


\begin{document}


\begin{CenterExample}
  \multido{\i=3+1}{6}{%
    \PstPolygon[PolyNbSides=\i]\hspace{5mm}}
  \PstPolygon[PolyNbSides=30]
\end{CenterExample}

\begin{CenterExample}
  \multido{\i=3+2}{6}{%
    \PstPolygon[PolyOffset=2,PolyNbSides=\i]\hspace{5mm}}
  \PstPolygon[PolyOffset=2,PolyNbSides=31]
\end{CenterExample}

\begin{CenterExample}
  \multido{\i=3+1}{7}{%
    \PstPolygon[PolyOffset=3,PolyNbSides=\i]\hspace{5mm}}
\end{CenterExample}

\begin{CenterExample}
  \multido{\i=5+1}{7}{%
    \PstPolygon[PolyOffset=4,PolyNbSides=\i]\hspace{5mm}}
\end{CenterExample}

\begin{CenterExample}
  \multido{\i=5+2}{7}{%
    \PstPolygon[PolyOffset=5,PolyNbSides=\i]\hspace{5mm}}
\end{CenterExample}

\clearpage
\begin{CenterExample}
  \multido{\i=5+2}{7}{%
    \PstPolygon[PolyOffset=7,PolyNbSides=\i]\hspace{5mm}}
\end{CenterExample}

\begin{CenterExample}
  \multido{\i=5+2}{7}{%
    \PstPolygon[PolyOffset=8,PolyNbSides=\i]\hspace{5mm}}
\end{CenterExample}

\begin{CenterExample}
  \multido{\i=1+1}{7}{%
    \PstPolygon[PolyOffset=\i,PolyNbSides=5]\hspace{5mm}}
\end{CenterExample}

\begin{CenterExample}
  \multido{\i=1+1}{7}{%
    \PstPolygon[PolyOffset=\i,PolyNbSides=7]\hspace{5mm}}
\end{CenterExample}

\begin{CenterExample}
  \multido{\i=5+1}{7}{%
    \PstPolygon[PolyCurves,PolyIntermediatePoint=0.1,PolyNbSides=\i]
    \hspace{5mm}}
\end{CenterExample}

\clearpage
\begin{CenterExample}
  \multido{\i=5+1}{7}{%
    \PstPolygon[PolyCurves,PolyIntermediatePoint=0.2,
                PolyOffset=2,PolyNbSides=\i]\hspace{5mm}}
\end{CenterExample}

\begin{CenterExample}
  \multido{\i=5+2}{7}{%
    \PstPolygon[PolyCurves,PolyIntermediatePoint=0.1,
                PolyOffset=3,PolyNbSides=\i]\hspace{5mm}}
\end{CenterExample}


\begin{CenterExample}
  \multido{\n=-1.4+0.5}{7}{%
    \PstPolygon[PolyNbSides=3,PolyOffset=2,PolyIntermediatePoint=\n]
    \hspace{5mm}}
\end{CenterExample}

\begin{CenterExample}
  \multido{\n=-1.4+0.5}{7}{%
    \PstPolygon[PolyNbSides=5,PolyOffset=2,PolyIntermediatePoint=\n]
    \hspace{5mm}}
\end{CenterExample}

\begin{CenterExample}
  \multido{\n=-1.4+0.5}{7}{%
    \PstPolygon[PolyNbSides=13,PolyOffset=2,PolyIntermediatePoint=\n]
    \hspace{5mm}}
\end{CenterExample}

\begin{CenterExample}
  \multido{\n=-1.4+0.5}{7}{%
    \PstPolygon[PolyNbSides=21,PolyOffset=2,PolyIntermediatePoint=\n]
    \hspace{5mm}}
\end{CenterExample}


\begin{CenterExample}
  \psset{unit=1.4,linewidth=0.001,PolyNbSides=72,PolyEpicycloid}
  \multido{\i=2+1}{4}{%
    % Epicycloid of factor 1 is cardioid and of factor 2 nephroid
    \PstPolygon[PolyOffset=\i]\hspace{5mm}}
\end{CenterExample}

\begin{SideBySideExample}[xrightmargin=5cm]
  % Epicycloid of factor 10
  \PstPolygon[unit=2,linewidth=0.003,
              PolyEpicycloid,PolyNbSides=72,PolyOffset=11]
\end{SideBySideExample}

\begin{SideBySideExample}[xrightmargin=5cm]
  % Epicycloid of factor 22
  \PstPolygon[unit=2,linewidth=0.003,
              PolyEpicycloid,PolyNbSides=72,PolyOffset=23]
\end{SideBySideExample}

\clearpage
\begin{CenterExample}
  \psset{unit=1.9,linewidth=0.001,PolyNbSides=72,PolyEpicycloid}
  \multido{\i=71+1}{3}{%
    \PstPolygon[PolyOffset=\i]\hspace{5mm}}
\end{CenterExample}

\fvset{xrightmargin=5cm} % 5cm reserved for the graphic

\begin{SideBySideExample}
  % Epicycloid of factor 100
  \PstPolygon[unit=2,linewidth=0.003,
              PolyEpicycloid,PolyNbSides=72,PolyOffset=101]
\end{SideBySideExample}

\begin{SideBySideExample}[xrightmargin=5cm]
  % Epicycloid of factor 153
  \PstPolygon[unit=2,linewidth=0.003,
              PolyEpicycloid,PolyNbSides=72,PolyOffset=154]
\end{SideBySideExample}

\clearpage


\begin{SideBySideExample}
  \providecommand{\PstPolygonNode}{%
    \psdots[dotsize=0.2,linecolor=cyan](1;\INode)}
  \PstPentagon[unit=2]
\end{SideBySideExample}

\begin{SideBySideExample}
  \providecommand{\PstPolygonNode}{%
    \rput{*0}(1.2;\INode){\small\the\multidocount}}
  \PstPolygon[unit=2,PolyNbSides=7,PolyOffset=2]
\end{SideBySideExample}

\begin{SideBySideExample}
  \providecommand{\PstPolygonNode}{%
    \rput*{*0}(1;\INode){\small\the\multidocount}}
  \PstHeptagon[unit=2,PolyOffset=2]
\end{SideBySideExample}

\begin{SideBySideExample}
  \newcounter{Letter}
  \providecommand{\PstPolygonNode}{%
    \setcounter{Letter}{\the\multidocount}%
    \rput*{*0}(1;\INode){\small\Alph{Letter}}}
  \PstHeptagon[unit=2,PolyOffset=3]
\end{SideBySideExample}

\begin{SideBySideExample}
  \providecommand{\PstPolygonNode}{%
    \SpecialCoor
    \degrees[3]
    \rput{0.5}(0.5;\INode){%
      \pspolygon*(0.5;0.5)(0.5;1.5)(0.5;2.5)}}
  \PstTriangle
\end{SideBySideExample}

\begin{SideBySideExample}
  \providecommand{\PstPolygonNode}{%
    \psdots[dotstyle=o,dotsize=0.2](1;\INode)
    \psline[linecolor=red]{->}(0.9;\INode)}
  \PstPolygon[unit=2,PolyNbSides=8]
\end{SideBySideExample}

\begin{SideBySideExample}
  \providecommand{\PstPolygonNode}{%
    \psline[linewidth=0.1mm,doubleline=true,
            linecolor=green]{<->}(0;0)(1;\INode)}
  \PstHexagon[unit=2]
\end{SideBySideExample}

\begin{SideBySideExample}
  \newbox{\Star}
  \savebox{\Star}{%
    \PstStarFive*[unit=0.15,linecolor=red]}
  \providecommand{\PstPolygonNode}{%
    \rput{*0}(1;\INode){\usebox{\Star}}}
  \shortstack{%
    \PstNonagon\\[5mm]
    \PstDodecagon[linestyle=none]}
\end{SideBySideExample}

\end{document}
\endinput
%%
%% End of file `tst-poly.tex'.
