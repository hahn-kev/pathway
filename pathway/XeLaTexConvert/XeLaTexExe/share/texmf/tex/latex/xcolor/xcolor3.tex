%%
%% This is file `xcolor3.tex',
%% generated with the docstrip utility.
%%
%% The original source files were:
%%
%% xcolor.dtx  (with options: `test3')
%% 
%% IMPORTANT NOTICE:
%% 
%% For the copyright see the source file.
%% 
%% Any modified versions of this file must be renamed
%% with new filenames distinct from xcolor3.tex.
%% 
%% For distribution of the original source see the terms
%% for copying and modification in the file xcolor.dtx.
%% 
%% This generated file may be distributed as long as the
%% original source files, as listed above, are part of the
%% same distribution. (The sources need not necessarily be
%% in the same archive or directory.)
%%
\ProvidesFile{xcolor3}
 [2007/01/21 v2.11 Color logging test (UK)]
%%
%% ----------------------------------------------------------------
%% Copyright (C) 2003-2007 by Dr. Uwe Kern <xcolor at ukern dot de>
%% ----------------------------------------------------------------
%%
\def\XCfileversion{v2.11}%
\def\XCfiledate{2007/01/21}%
\listfiles
\documentclass[a4paper]{article}
\usepackage[showerrors,table,dvipsnames*,hyperref]{xcolor}[2005/12/21]
\usepackage[margin=2.25cm]{geometry}
\usepackage
 [\GinDriver,hyperindex=false,bookmarks,bookmarksopen,bookmarksopenlevel=1,%
  pdftitle={xcolor3 \XCfileversion{} (\XCfiledate)},pdfauthor={Dr. Uwe Kern},%
  pdfsubject={Color extensions for LaTeX and pdfLaTeX},%
  pdfkeywords={xcolor,color,colour,model,tint,tone,shade,harmony,spot,latex,pdftex,dvips,%
   conversion,blend,mix,mask,separation,rgb,cmy,cmyk,hsb,gray,html,wave,thsb,wheel}]{hyperref}

\tracingcolors=4
%%\tracingcolors=3
%%\tracingcolors=2
%%\tracingcolors=1
%%\tracingcolors=0

\parindent0pt
\pagecolor{gray!25}

\definecolors{JungleGreen,DarkOrchid}

\begin{document}
\title{Color extensions with the \textsf{xcolor} package --- various examples}
\author{\href{mailto:xcolor@ukern.de}{\fboxrule0pt\fboxsep2pt\fbox{Dr. Uwe Kern}}}
\date{\XCfileversion{} (\XCfiledate)
\thanks{This file (\texttt{\jobname.tex}) is part of the \textsf{xcolor} distribution which can be downloaded from the CTAN mirrors \texttt{\href{http://www.ctan.org/tex-archive/macros/latex/contrib/xcolor/}{CTAN/macros/latex/contrib/xcolor/}} or the homepage \texttt{\href{http://www.ukern.de/tex/xcolor.html}{www.ukern.de/tex/xcolor.html}}. Please send error reports and suggestions for improvements to \texttt{\href{mailto:xcolor@ukern.de}{xcolor@ukern.de}}.}}
\maketitle

The purpose of this file is to demonstrate a variety of capabilities including the logging facilities of the \textsf{xcolor} package.
By playing around with different values of \texttt{\string\tracingcolors}, one can observe the different behavior in the \texttt{log} file.

\section{Predefined colors}

\begingroup
\small\sffamily
\rowcolors1{}{}
\begin{testcolors}[rgb,cmyk,hsb,HTML,gray]
\testcolor{red}
\testcolor{green}
\testcolor{blue}
\testcolor{cyan}
\testcolor{magenta}
\testcolor{yellow}
\testcolor{orange}
\testcolor{violet}
\testcolor{purple}
\testcolor{brown}
\testcolor{pink}
\testcolor{olive}
\testcolor{black}
\testcolor{darkgray}
\testcolor{gray}
\testcolor{lightgray}
\testcolor{white}
\noalign{\medskip}\hline\noalign{\medskip}
\testcolor{-red}
\testcolor{-green}
\testcolor{-blue}
\testcolor{-cyan}
\testcolor{-magenta}
\testcolor{-yellow}
\testcolor{-orange}
\testcolor{-violet}
\testcolor{-purple}
\testcolor{-brown}
\testcolor{-pink}
\testcolor{-olive}
\testcolor{-black}
\testcolor{-darkgray}
\testcolor{-gray}
\testcolor{-lightgray}
\testcolor{-white}
\noalign{\medskip}\hline\noalign{\medskip}
\testcolor{JungleGreen}
\testcolor{DarkOrchid}
\noalign{\medskip}\hline\noalign{\medskip}
\testcolor{-JungleGreen}
\testcolor{-DarkOrchid}
\end{testcolors}
\endgroup

\vfill

\clearpage
\pagecolor{white}

\section{Color definition and application}

\providecolor{dummy}{rgb}{.6,.5,.4}
\definecolor{dummy}{rgb}{.6,.5,.4}
\providecolor{dummy}{rgb}{.6,.5,.4}
\hbox{\textcolor{dummy}{Test with \texttt{\string\definecolor}}}

\bigskip

Comma-separated and space-separated definitions:

\definecolor{c1}{rgb}{.7,.6,.5}
\definecolor{c2}{rgb}{.7 .6 .5}
\colorlet{c1a}{c1}
\colorlet{c2a}{c2}

\textcolor{c1}{identical} =
\textcolor{c2}{identical} =
\textcolor{c1a}{identical} =
\textcolor{c2a}{identical} =
\textcolor[rgb]{.7,.6,.5}{identical} =
\textcolor[rgb]{.7 .6 .5}{identical} =
\textcolor{rgb,10:red,7;green,6;blue,5}{identical}
\textcolor{rgb,15:red,10.5;green,9;blue,7.5}{identical}

\medskip

\begingroup
\sffamily
\begin{testcolors}
\testcolor{c1}
\testcolor{c2}
\testcolor{c1a}
\testcolor{c2a}
\testcolor[rgb]{.7,.6,.5}
\testcolor[rgb]{.7 .6 .5}
\testcolor{rgb,10:red,7;green,6;blue,5}
\testcolor{rgb,15:red,10.5;green,9;blue,7.5}
\end{testcolors}
\endgroup

\bigskip

\textcolor{rgb:red!50,4;green!25,2}{Another extended color expression (rgb:red!50,4;green!25,2)}.

\bigskip

\begingroup
\color{black}
Test with named colors:\par
\color{blue}
Test: \textcolor[named]{JungleGreen}{JungleGreen};
Test: \textcolor{JungleGreen}{JungleGreen};
Test: \textcolor{JungleGreen!50!DarkOrchid}{JungleGreen!50!DarkOrchid};
Test: \textcolor{green!50!red}{green!50!red}.
\endgroup

\bigskip

{\color[rgb]{.4,.5,.6}Test with \texttt{\string\color}}

\bigskip
Current color application:\par
\def\test{current, \textcolor{.!50}{50\%}, \textcolor{-.}{complement},
          \textcolor{yellow!50!.}{mix}}
\textcolor{blue}{\test} and \textcolor{red}{\test},\par
\def\Test{\color{.!80}Test}
\textcolor{blue}{\Test\Test\Test\Test\Test} and
\textcolor{red}{\Test\Test\Test\Test\Test}.

\bigskip
Current color test with \texttt{\string\definecolorseries}:\par
\begingroup
\color{blue}
\definecolorseries{foo}{rgb}{last}{.}{-.}
\resetcolorseries[5]{foo}
\def\test{\hbox to 1em{{\color{foo!!+}\vrule width 1em height 1.5ex}}}
Test\test\test\test\test\test\test Test

\resetcolorseries[5]{foo}
\def\test{\hbox to 1em{{\color{foo!!++}\vrule width 1em height 1.5ex}}}
Test\test\test\test\test\test\test Test

\resetcolorseries[5]{foo}
\def\test{\hbox to 1em{{\color{foo!![2]}\vrule width 1em height 1.5ex}}}
Test\test\test\test\test\test\test Test

\endgroup

\section{Color in tables}

\rowcolors[\hline]{1}{green!25}{yellow!50}
\begin{tabular}{ll}
test & row \number\rownum\\
test & row \number\rownum\\
\rowcolor{blue!25}
test & row \number\rownum\\
test & row \number\rownum\\
\hiderowcolors
test & row \number\rownum\\
test & row \number\rownum\\
\showrowcolors
test & row \number\rownum\\
test & row \number\rownum\\
\multicolumn{1}%
 {>{\columncolor{red!12}}l}{test} & row \number\rownum\\
\end{tabular}

\section{Color information}

Type test:
\makeatletter
\@namedef{\string\color@foo1}{foo1{}{}{}{}}\edef\tempa{\XC@type{foo1}}\tempa
\@namedef{\string\color@foo2}{\xcolor@{foo2}{}{}{}}\edef\tempb{\XC@type{foo2}}\tempb
\@namedef{\string\color@foo3}{\xcolor@{}{foo3}{}{}}\edef\tempc{\XC@type{foo3}}\tempc
\@namedef{\string\color@foo4}{\xcolor@{}{}{foo4}{}}\edef\tempd{\XC@type{foo4}}\tempd
\makeatother

\end{document}
\endinput
%%
%% End of file `xcolor3.tex'.
